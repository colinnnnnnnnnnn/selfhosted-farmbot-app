\chapter{INTRODUCTION}

\section{Context and Background}
Modern agriculture stands at a critical juncture, facing unprecedented challenges in optimizing resource usage and improving productivity, particularly within urban and small-scale farming environments \cite{agtech2023}. The rapid growth of urban populations, combined with increasing food security concerns, has created an urgent need for innovative agricultural solutions. The emergence of precision agriculture and automated farming systems presents promising opportunities to address these challenges while promoting sustainable practices.

This project focuses on the development of a FarmBot system, an innovative solution that combines advanced robotics and Internet of Things (IoT) technology to revolutionize small-scale farming \cite{farmbot2020}. By integrating automation, precision control, and data-driven decision-making, FarmBot aims to transform how small-scale and urban farming operations manage their resources and optimize their productivity.

\section{Motivation and Relevance}
The motivation for this project emerges from the convergence of several critical factors shaping modern agriculture. The increasing urbanization of global populations has created an urgent need for efficient urban farming solutions, as cities must become more self-sufficient in food production. Traditional agricultural models, centered around large, distant farms, are becoming increasingly unsustainable for meeting local food needs in urban environments.

The growing demand for automation in small-scale agriculture reflects the need to optimize resource usage while maintaining high productivity levels. Small-scale farmers face significant challenges in competing with larger operations, and automation offers a path to improved efficiency and sustainability. The rising importance of resource optimization in farming stems from both environmental concerns and economic necessities, as water scarcity and energy costs continue to impact agricultural operations.

The accessibility of precision farming technologies represents another crucial motivation. While sophisticated farming automation exists for large-scale operations, small-scale farmers often lack access to appropriately scaled and affordable solutions. FarmBot addresses this gap by providing a scalable, cost-effective automation platform that meets the specific needs of small-scale operations.

\section{Potential Impact}
The FarmBot project's potential impact extends across multiple stakeholder groups, each benefiting from different aspects of the system. For small-scale farmers, the primary impact lies in operational efficiency improvements through automated precision farming techniques. The technology enables more accurate planting, optimal resource usage, and reduced labor requirements, helping these farmers remain competitive in an increasingly challenging market.

Urban farmers gain particular value from FarmBot's ability to maximize productivity in limited spaces \cite{urban_farming2022}. The system's compact design and efficient resource management capabilities make it ideal for urban agricultural initiatives, enabling sustainable food production within city environments. This impact extends beyond individual farmers to contribute to broader urban food security and sustainability goals.

Educational institutions benefit from FarmBot's role as a practical teaching platform for modern agricultural concepts. The system provides hands-on experience with automation, sustainability, and precision farming techniques, helping prepare students for the future of agriculture. Research organizations gain valuable data collection and analysis capabilities, supporting continued innovation in agricultural practices and technologies.



