\chapter{REQUIREMENTS SPECIFICATION}

This chapter outlines the comprehensive requirements specification for the FarmBot system, including user stories that capture stakeholder needs, system requirements that detail technical specifications, and system components that form the architecture.

\section{User Stories}

The following user stories capture the key requirements from different stakeholders' perspectives:

\begin{longtable}{|p{0.12\textwidth}|p{0.48\textwidth}|p{0.15\textwidth}|p{0.25\textwidth}|}
\hline
\textbf{ID} & \textbf{Story} & \textbf{Importance} & \textbf{Additional Notes} \\
\hline
\endhead
US-001 & As a farmer, I want automatic pest detection through FarmBot cameras so that I can reduce crop losses without constant monitoring. & Must Have & AI must identify >90\% of common pests. \\
\hline
US-002 & As a farmer, I want FarmBot to recommend irrigation schedules based on soil moisture so that I can reduce unnecessary water use. & Must Have & Apply MoSCoW prioritization. \\
\hline
US-003 & As a developer, I want modular APIs for integrating multiple weather providers so that FarmBot is more flexible. & Should Have & API must support at least 3 main providers. \\
\hline
US-004 & As a policymaker, I want reporting dashboards from FarmBot so that I can track sustainability indicators. & Should Have & Dashboard aligned with national agri KPIs. \\
\hline
US-005 & As a cooperative manager, I want multi-user access in FarmBot so that members can coordinate tasks together. & Must Have & Login with roles and permissions. \\
\hline
US-006 & As a researcher, I want anonymized FarmBot datasets so that I can study crop efficiency without privacy risks. & Must Have & Must comply with GDPR. \\
\hline
US-007 & As an agronomist, I want FarmBot soil nutrient analytics so that I can recommend better crop rotation strategies. & Could Have & Reports must have <5\% error margin. \\
\hline
US-008 & As a visually impaired user, I want voice-guided FarmBot instructions so that I can interact without a screen. & Must Have & Screen-reader compatibility and voice navigation. \\
\hline
US-009 & As a farmer, I want FarmBot to generate seasonal yield forecasts so that I can plan sales better. & Could Have & Forecast validated with 3 past seasons. \\
\hline
US-010 & As an extension officer, I want FarmBot tutorials in local languages so that farmers can clearly understand the instructions. & Should Have & At least 3 language versions available. \\
\hline
US-011 & As a government official, I want FarmBot to generate compliance reports on water and pesticide use so that I can ensure farms meet environmental regulations. & Must Have & Reports must align with EU Farm to Fork and local legislation. \\
\hline
\end{longtable}

\section{System Requirements}

Based on the user stories, the following system requirements were derived:

\begin{longtable}{|p{0.12\textwidth}|p{0.48\textwidth}|p{0.15\textwidth}|p{0.25\textwidth}|}
\hline
\textbf{ID} & \textbf{Requirement} & \textbf{Satisfy Story} & \textbf{Additional Notes} \\
\hline
\endhead
SYR-001 & The system must include visual recognition algorithms for pest detection. & US-001 & AI trained on local datasets. \\
\hline
SYR-002 & Soil sensors must provide data for generating automatic irrigation schedules. & US-002 & Data collected at 30-minute intervals. \\
\hline
SYR-003 & The FarmBot API must support integration with at least 3 different weather providers. & US-003 & Support for OpenWeather, ECMWF, and a local provider. \\
\hline
SYR-004 & Dashboards must include reports on water use, energy consumption, and crop yield. & US-004 & Export available in PDF and CSV. \\
\hline
SYR-005 & The system must support multi-user accounts with role-based access (admin, technician, operator). & US-005 & Granular permission control. \\
\hline
SYR-006 & FarmBot data must be anonymized before being shared with third parties. & US-006 & Must comply with GDPR and ISO/IEC 20889. \\
\hline
SYR-007 & Soil sensors must report NPK levels with <5\% error margin. & US-007 & Results validated through independent testing. \\
\hline
SYR-008 & The system must include voice guidance and screen-reader compatibility. & US-008 & Tested with NVDA and JAWS. \\
\hline
SYR-009 & The forecasting module must calculate yields based on historical and real-time data. & US-009 & Forecasts tested on the last 3 agricultural seasons. \\
\hline
SYR-010 & FarmBot tutorials must be available in at least 3 regional languages. & US-010 & Translations verified by native speakers. \\
\hline
SYR-011 & FarmBot must generate automated compliance reports on pesticide and water usage. & US-011 & Reports formatted to meet EU Farm to Fork and local requirements. \\
\hline
\end{longtable}

\section{System Components}

The FarmBot system consists of the following major components:

\begin{longtable}{|p{0.25\textwidth}|p{0.35\textwidth}|p{0.4\textwidth}|}
\hline
\textbf{Component} & \textbf{Description} & \textbf{Additional Notes} \\
\hline
\endhead
FarmBot Web App & Web interface for farm design, control, and photo viewing & Hosted at my.farmbot.io; communicates via MQTT \\
\hline
Django Backend & Handles user authentication, API logic, and token management & Built with Django; interfaces with DB and FarmBot API \\
\hline
MQTT Gateway & Real-time message broker between frontend and FarmBot device & Uses MQTT protocol; bridges cloud and hardware \\
\hline
OpenFarm.cc & External crop database providing plant information & Enriches farm planning via API integration \\
\hline
FarmBot Configurator & Setup tool for WiFi and web app credentials & Used during initial device configuration \\
\hline
Raspberry Pi Controller & Central hardware unit running FarmBot OS; executes commands and logs & Interfaces with Arduino, webcam, and MQTT; controls logic and data flow \\
\hline
Arduino/RAMPS Board & Microcontroller board controlling motors, sensors, and tools & Receives G-CODE from Raspberry Pi; runs firmware for physical actuation \\
\hline
Stepper Motors & Drive the FarmBot's X, Y, Z axis movement & Controlled via Arduino; essential for precision positioning \\
\hline
Rotary Encoders & Provide feedback on motor rotation and position & Improve accuracy of movement; connected to stepper motors \\
\hline
Watering Tool & Dispenses water to plants based on duration and location & Activated via G-CODE; mounted on tool head \\
\hline
Soil Sensor & Measures soil moisture and other environmental data & Sends sensor data to Raspberry Pi for logging and decision-making \\
\hline
Webcam / USB Camera & Captures images of the farm environment & Connected to Raspberry Pi; used for visual monitoring and documentation \\
\hline
Camera Mount & Physical holder for the webcam on the tool head & Ensures stable image capture; adjustable for angle and focus \\
\hline
Tool Bay & Storage and switching station for FarmBot tools & Allows automatic tool changes; controlled via Raspberry Pi and Arduino \\
\hline
Power Supply Unit & Provides electrical power to all FarmBot components & Converts AC to DC; includes voltage regulation and safety features \\
\hline
WiFi Module & Enables network connectivity for Raspberry Pi & Required for cloud communication and remote control \\
\hline
Frame \& Gantry System & Structural hardware for movement across the farm bed & Supports X/Y/Z axis; includes rails, belts, and pulleys \\
\hline
\end{longtable}

Each component plays a crucial role in the overall functionality of the FarmBot system, working together to deliver the features and capabilities defined in the user stories and system requirements. The components are designed with modularity in mind, allowing for easy maintenance, upgrades, and scalability of the system.
