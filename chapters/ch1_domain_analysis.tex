\chapter{DOMAIN ANALYSIS}

\section{Problem Overview}
The agricultural sector, particularly in urban and small-scale environments, faces significant challenges that impact food production efficiency and sustainability \cite{agtech2023}. Modern agriculture is undergoing a profound transformation driven by increasing urbanization, growing population demands, and mounting environmental pressures. In urban areas, traditional farming approaches become increasingly inadequate as space constraints and resource limitations create unique challenges for food production.

Small-scale farmers face multiple interconnected challenges that affect their productivity and sustainability. Limited space utilization in urban environments forces farmers to maximize yield from minimal area, often without proper tools or systems to optimize their operations. The lack of efficient resource management systems leads to significant water wastage and increased operational costs, making it difficult for small-scale operations to remain economically viable.

The persistence of manual, time-consuming farming operations presents another significant barrier. Without proper automation, farmers spend excessive time on routine tasks that could be optimized through technology. The absence of precise monitoring and control systems makes it challenging to maintain optimal growing conditions and respond quickly to environmental changes or crop needs \cite{urban_farming2022}.

\section{Target Audience}
Our comprehensive market research, conducted through detailed surveys and in-depth interviews with 50 potential users, reveals a diverse set of stakeholders with distinct needs and challenges. Urban farmers emerge as our primary user segment, representing 40% of respondents. These individuals typically operate in confined spaces such as rooftops, balconies, or small community gardens. Their primary challenges include maximizing productivity in limited spaces while maintaining sustainable practices and managing resources efficiently.

Small-scale commercial farmers constitute 30% of our target audience. These operators seek to balance traditional farming knowledge with modern automation to improve their competitiveness. They face significant challenges in managing operational costs while trying to scale their production to meet growing market demands.

Educational institutions, comprising 20% of our survey respondents, represent a unique segment that requires both practical and theoretical applications. These organizations seek tools that can effectively demonstrate modern agricultural practices while providing hands-on learning experiences for students. Their primary challenge lies in accessing affordable, comprehensive farming technology that serves both educational and practical purposes.

Research organizations, while representing 10% of respondents, provide crucial insights into agricultural innovation. These institutions require sophisticated data collection and analysis capabilities to conduct meaningful research on crop yields, resource efficiency, and sustainable farming practices.

\section{Solution Concept}
The FarmBot system addresses these challenges through an innovative, integrated approach that combines advanced automation with intelligent resource management \cite{automation2023}. At its core, the system provides comprehensive automation of essential farming tasks, including precision planting, targeted irrigation, and systematic maintenance. This integration significantly reduces manual labor requirements while improving overall resource utilization.

The platform incorporates sophisticated monitoring capabilities through a network of environmental sensors that track conditions, soil moisture, and plant health in real-time. This data-driven approach enables farmers to make informed decisions about resource allocation and crop management, leading to improved yields and reduced waste.

Resource optimization stands as a central pillar of the FarmBot system. Through intelligent water usage tracking, energy consumption monitoring, and space utilization analysis, the system helps users maximize efficiency while minimizing environmental impact. The platform's modular design allows for customization based on specific needs and constraints, making it adaptable to various farming scenarios.

\section{Market Research}
Our analysis of existing agricultural automation solutions reveals significant gaps in the current market \cite{iot_agriculture2023}. While traditional automation systems typically focus on single-task operations, FarmBot offers a comprehensive, integrated approach that addresses multiple aspects of farming simultaneously. This holistic approach sets our solution apart in the market and provides unique value to users.

\begin{table}[htbp]
\centering
\caption{Market Analysis of Agricultural Automation Solutions}
\begin{tabular}{|p{3cm}|p{5cm}|p{5cm}|}
\hline
\textbf{Feature} & \textbf{Current Market Solutions} & \textbf{FarmBot Innovation} \\
\hline
Automation & Limited to isolated tasks & Comprehensive integrated system \\
\hline
User Interface & Technical expertise required & Intuitive, user-centric design \\
\hline
Data Analytics & Basic metrics tracking & Advanced predictive analytics \\
\hline
Adaptability & Fixed configurations & Modular, customizable design \\
\hline
Cost Structure & High initial investment & Scalable, accessible pricing \\
\hline
\end{tabular}
\end{table}

\section{User Stories and Requirements}

\begin{longtable}{|p{0.1\textwidth}|p{0.5\textwidth}|p{0.15\textwidth}|p{0.15\textwidth}|}
\caption{FarmBot System User Stories and Requirements} \label{tab:user_stories} \\
\hline
\textbf{ID} & \textbf{User Story} & \textbf{Priority} & \textbf{Status} \\
\hline
\endfirsthead


\endhead


\endfoot


\hline
\endlastfoot
US-001 & As an urban farmer, I need to manage my garden layout through an intuitive visual interface to optimize space utilization & High & Planned \\
\hline
US-002 & As a small-scale farmer, I want automated alerts about plant health and resource usage to prevent crop losses & High & In Progress \\
\hline
US-003 & As an educator, I need detailed analytics and visualization tools to demonstrate farming concepts to students & Medium & Planned \\
\end{longtable}

\begin{longtable}{|p{0.1\textwidth}|p{0.5\textwidth}|p{0.15\textwidth}|p{0.15\textwidth}|}
\multicolumn{4}{r}{\textbf{\normalsize{Continuation of table 2.2}}} \\
\hline
\textbf{ID} & \textbf{User Story} & \textbf{Priority} & \textbf{Status} \\
\hline
US-004 & As a researcher, I want to collect and export comprehensive growing data for analysis & Medium & To Do \\
\hline
US-005 & As a farmer, I need automated watering based on soil moisture levels to optimize water usage & High & In Progress \\
\hline
US-006 & As a user, I want to monitor my farm remotely through a mobile application & High & Planned \\
\hline
US-007 & As an administrator, I need to manage multiple user accounts and access levels & Medium & To Do \\
\hline
US-008 & As a maintenance technician, I need diagnostic tools to quickly identify and resolve system issues & Medium & Planned \\
\hline
\end{longtable}

Modern agriculture is undergoing significant transformation due to rapid population growth, increased urbanization, and rising pressure on natural resources. These global dynamics create new constraints and force the agricultural sector to seek innovative and sustainable methods of production.

\subsection{Problem Identification}

Small-scale and urban farmers represent one of the most vulnerable categories, as they often operate with limited land, restricted access to technology, and high reliance on manual labor. Traditional farming methods, although familiar and widely used, are becoming increasingly inefficient in the face of such challenges.

Stakeholder analysis reveals several key perspectives:
\begin{itemize}
\item Small-scale and urban farmers highlight resource inefficiency issues
\item Consumers emphasize growing demand for healthy, local, sustainable products
\item Authorities advocate for environmentally friendly farming approaches
\item Technology providers often design inaccessible solutions
\end{itemize}

\subsection{Key Challenges}

The situation presents several persistent challenges:
\begin{itemize}
\item Resource inefficiency in traditional methods
\item Shortage of available labor affecting productivity
\item Limited access to advanced technologies
\item Space constraints in urban environments
\item High initial investment costs and technical complexity
\end{itemize}

\subsection{Problem Statement}

The agricultural practices and automation solutions currently available do not adequately meet the needs of small-scale and urban farmers. Limitations in resource management, labor availability, and technology adoption reduce overall efficiency and restrict the potential for sustainable development.

\section{Proposed Solution}

The proposed solution is an open-source automated farming system that integrates principles of precision agriculture with an affordable and adaptable design. FarmBot is specifically intended to provide small-scale and urban farmers with the opportunity to use technology that ensures:
\begin{itemize}
\item More efficient resource management
\item Reduced dependence on manual labor
\item Establishment of sustainable food production systems
\end{itemize}

\section{System Interaction and Use Cases}

The FarmBot system features:
\begin{itemize}
\item Intuitive web and mobile interfaces
\item Autonomous operation for essential agricultural tasks
\item Modular construction for different environments
\item Sensor-based data collection and analysis
\item Automatic reporting and monitoring
\end{itemize}


